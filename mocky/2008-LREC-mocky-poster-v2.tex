\documentclass[largefonts]{sciposter}


\usepackage{standard}
\usepackage{multicol}
\usepackage{url}

\title{Designing and Evaluating \\a Russian Tagset}

\author{Serge Sharoff$^{\star}$,  Mikhail Kopotev$^{\ast}$,  Toma\v{z} Erjavec$^{\dagger}$,  \\
Anna Feldman$^{\ddagger}$, Dagmar Divjak$^{\diamond}$}
 
\institute{
$^{\star}$University of Leeds, $^{\ast}$University of Helsinki,$^{\dagger}$Jo\v{z}ef Stefan Institute,  \\
$^{\ddagger}$Montclair State University,$^{\diamond}$University of Sheffield
}

\email{s.sharoff@leeds.ac.uk, mihail.kopotev@helsinki.fi, tomaz.erjavec@ijs.si, feldmana@mail.montclair.edu, d.divjak@shef.ac.uk}  

\leftlogo[1.3]{/home/serge/logo-leeds-ijs}
\rightlogo[1.3]{/home/serge/logo-joint-helsinki-shef-msu}


\begin{document}

\conference{{\bf LREC 2008}, 6th Language Resources and Evaluation Conference, 28-29-30 May 2008, Marrakech (Morocco). }

\maketitle

\begin{multicols}{2}

\section{Existing research}

\begin{itemize}
\item MT in 1950s -- hand-crafted ad-hoc rules
\item Zalizniak (1977) -- formalisation of Russian morphology 
\item Mikheev, Segalovich, Sokirko, Starostin -- various implementations of Russian analysis and synthesis
\item Sokirko, Feldman -- experiments with statistical taggers, but no resources
\end{itemize}

\section{Problems with Russian}

\paragraph{Free word order and very rich morphology} it is morphology
that plays a crucial role in signaling the syntactic relationships.

\paragraph{Low number of morpheme forms} The same form in different contexts can be interpreted in different ways:

\begin{examples}
\item \gll \R{анализ} \R{структуры}
  analysis structure$_{gen,sg}$
  \glt `analysis of the structure'
  \glend
\item \gll \R{в} \R{эти} \R{структуры}
 in these structure$_{acc,pl}$
  \glt `into these structures'
  \glend
\item \gll \R{эти} \R{структуры} \R{привлечены} \R{к}
 these structure$_{nom,pl}$ involve$_{part,pass,perf,past,pl}$ to
  \glt `these structures are involved in...'
  \glend
\end{examples}

\paragraph{No tagset exists} Zalizniak (1977) uses a system of categories (case, gender, number, etc), but not a tagset, \eg \code{NN, NNS, NP}, etc.  Analysis applications do not disambiguate, \eg\R{структуры} $\rightarrow gen,sg;acc,pl;nom,pl$.

\paragraph{Problems with disambiguating large tagsets}  about 50 tags for English vs. 500-2000 tags for Russian.

\section{Tagset principles}

\paragraph{MULTEXT-East (MTE)} A freely available multilingual dataset for
language engineering research and development. The resources cover a
large number of mainly Central and Eastern European languages.

\paragraph{MTE morphosyntactic specifications} Main morphosyntactic categories (nouns, verbs, pronouns, \ldots) and their
allowed attribute-value pairs.  Feature-structures describing
morphosyntactic properties of words can be mapped to compact strings,
morphosyntactic descriptions (MSDs). 

\begin{figure}[h]

  \begin{center}
    \includegraphics[width=0.7\columnwidth]{mapping}
  \end{center}
    \code{<span lex='\R{структура}'  class='S=f,gen,inan,sg'>\R{структуры}</span>}

    $\rightarrow$ \R{структуры} \code{Ncfsgn} \R{структура}

    $\rightarrow$ \code{Noun, Type = common, Gender = feminine, Number = singular, Case = genitive, Animate = no} 

\end{figure}

\section{Properties of the tagset}
\label{sec:properties}

\begin{itemize}
\item the balance between parameters important for linguists and the
  possibility of their automatical detection;
\item the availability of features in existing corpora that can be used
  for training;
\item the possibility to share the tagset with other Slavonic languages to
create, in perspective, a common Slavonic morphological tagset.
\end{itemize}

\paragraph{The resulting tagset} 12 main categories: noun, verb, adjective, pronoun, adverb, adposition,
conjunction, numeral, particle, interjection, abbreviation, and residual, having 0-10 attributes each. In total giving 156 attribute-value pairs

\section{Evaluation}

\begin{itemize}
\item The disambiguated portion of the RNC (about 5 million words)
\item Three statistical POS taggers: TnT, TreeTagger and SVM Tagger
\item 10\% of the corpus held out for testing the performance of the taggers
\item an experiment with ten second year British
  students of Russian revealed that intermediate level students  are not able to
spot the errors produced by the taggers: they too seem
  to analyze forms in isolation.  
\item Common problems: accusative vs.~nominative; genitive singular vs. nominative/accusative plural for nouns; genitive singular vs.~instrumental for feminine adjectives 
\end{itemize}

\begin{table}
  \begin{center}
    \begin{tabular}{lllll}
      \hline
      Accuracy & Overall & Known W & Unknown W\\
      \hline
      TnT  & 95.28\% & 96.27\% & 66.64\%\\
      TT  & 93.50\% & 94.33\% & 62.44\%\\
      SVMTool  & 92.24\% & 93.26\% & 54.28\%\\
      \hline
    \end{tabular}

    Overall accuracy of TnT, TT, and SVMTool, full tagset

\vspace{2cm}

    %%   \end{center}
    %%   \caption{}
    %% \label{tab:performance}
    %% \end{table}

    %% \begin{table}[h]
    %% \begin{center}
    \begin{tabular}{lrr}
      \hline
      % & \multicolumn{2}{c}{TnT} }\\
      & known & unknown \\
      \hline
      \hline
      full tag & 90.99 & 56.05 \\
      \hline \hline

      1: category & 99.02 & 93.61  \\
      2: type & 98.42& 86.00  \\
      3: gender & 97.51 & 77.23 \\
      4: number & 97.89 & 89.26  \\
      5: case & 93.03 & 80.23 \\
      %6:animate & 96.84& 69.04\\
      %7:case2 & 99.02 & 93.61\\
      \hline
    \end{tabular}

    Accuracy of TnT  on nouns in \% (full tagset).

\vspace{2cm}

    \begin{tabular}{lrr}
      \hline
      % & \multicolumn{2}{c}{TnT} &\multicolumn{2}{c}{TT}\\
      & known & unknown \\
      \hline \hline
      full tag & 96.34 & 73.12 \\
      \hline \hline

      1: category & 99.00 & 93.74 \\
      2: type & 99.00& 93.74 \\
      3: vform & 98.61 & 91.44 \\
      4: tense & 97.69 & 84.10  \\
      5: person & 98.93 & 93.33 \\
      6: number & 98.80 & 93.42  \\
      7: gender & 98.95 & 93.57 \\
      8: voice & 98.89 & 93.01 \\
      9: definiteness & 98.97 & 93.60 \\
      10: aspect & 96.93 & 75.23 \\
      11: case & 98.98 & 93.68  \\
      \hline
    \end{tabular}

    Accuracy of TnT on verbs in \% (full tagset).

\vspace{2cm}

      \begin{tabular}{lrr}
        \hline
        %$ & \multicolumn{2}{c}{TnT} &\multicolumn{2}{c}{TT}\\
        & known & unknown \\
        \hline \hline
        full tag & 89.13 & 80.51 \\
        \hline \hline

        Tag slot & &\\
        \hline \hline

        1: category & 97.25 & 91.72 \\
        2: type & 97.25& 91.72 \\
        3: degree & 97.24 & 91.72  \\
        4: gender & 95.67 & 89.77  \\
        5: number & 97.00 & 90.98 \\
        6: case & 90.54 & 84.37 \\
        %7:definiteness & & 87.53 & & \\
        \hline
      \end{tabular}

      Accuracy of TnT on adjectives in \% (full tagset).
  \end{center}
\end{table}


Resources: \url{http://corpus.leeds.ac.uk/mocky/}

\end{multicols}

\end{document}

